%\documentclass[10pt,notes]{beamer}       % print frame + notes
%\documentclass[10pt, notes=only]{beamer}   % only notes
\documentclass[11pt]{beamer}              % only frames

%%%%%% IF YOU WOULD LIKE TO CREATE LECTURE NOTES COMMENT OUT THE FOlLOWING TWO LINES
%\usepackage{pgfpages}
%\setbeameroption{show notes on second screen=bottom} % Both

\usepackage{graphicx}
\DeclareGraphicsExtensions{.pdf,.png,.jpg}
\usepackage{color}
\usetheme{winslab}
\usepackage[utf8]{inputenc}
\usepackage[english]{babel}
\usepackage{amsmath}
\usepackage{amsfonts}
\usepackage{amssymb}




\usepackage{algorithm2e,algorithmicx,algpseudocode}
\algnewcommand\Input{\item[\textbf{Input:}]}%
\algnewcommand\Output{\item[\textbf{Output:}]}%
\newcommand\tab[1][1cm]{\hspace*{#1}}

\algnewcommand{\Implement}[2]{\item[\textbf{Implements:}] #1 \textbf{Instance}: #2}%
\algnewcommand{\Use}[2]{\item[\textbf{Uses:}] #1 \textbf{Instance}: #2}%
\algnewcommand{\Trigger}[1]{\Statex{\textbf{Trigger:} (#1)}}%
\algnewcommand{\Events}[1]{\item[\textbf{Events:}] #1}%
\algnewcommand{\Need}[1]{\item[\textbf{Needs:}] #1}%
\algnewcommand{\Event}[2]{\Statex \item[\textbf{On#1:}](#2) \textbf{do}}%
\algnewcommand{\Trig}[3]{\State \textbf{Trigger}  #1.#2 (#3) }%
\def\true{\textbf{T}}
\def\false{\textbf{F}}


\author[Ahmet Kürşad Şaşmaz]{Ahmet Kürşad Şaşmaz\\\href{mailto:kursad.sasmaz@metu.edu.tr}{kursad.sasmaz@metu.edu.tr}}
%\author[J.\,Doe \& J.\,Doe]
%{%
%  \texorpdfstring{
%    \begin{columns}%[onlytextwidth]
%      \column{.45\linewidth}
%      \centering
%      John Doe\\
%      \href{mailto:john@example.com}{john@example.com}
%      \column{.45\linewidth}
%      \centering
%      Jane Doe\\
%      \href{mailto:jane.doe@example.com}{jane.doe@example.com}
%    \end{columns}
%  }
%  {John Doe \& Jane Doe}
%}

\title[Raymond's Algorithm]{Raymond's Algorithm}
\subtitle[Mutual Exclusion in Distributed Systems]{Mutual Exclusion in Distributed Systems}
%\date{} 

\begin{document}

\begin{frame}[plain]
\titlepage
\note{In this talk, I will present .... Please answer the following questions:
\begin{enumerate}
\item Why are you giving presentation?
\item What is your desired outcome?
\item What does the audience already know  about your topic?
\item What are their interests?
\item What are key points?
\end{enumerate}
}
\end{frame}

\begin{frame}[label=toc]
    \frametitle{Outline of the Presentation}
    \tableofcontents[subsubsectionstyle=hide]
\note{ The possible outline of a talk can be as follows.
\begin{enumerate}
\item Outline 
\item Problem and background
\item Design and methods
\item Major findings
\item Conclusion and recommendations 
\end{enumerate} Please select meaningful section headings that represent the content rather than generic terms such as ``the problem''. Employ top-down structure: from general to more specific.
}
\end{frame}
%
%\part{This the First Part of the Presentation}
%\begin{frame}
%        \partpage
%\end{frame}
%
\section{The Problem}
%\begin{frame}
%        \sectionpage
%\end{frame}

\begin{frame}{The Problem}
\framesubtitle{Selecting ownere of a resource without any conflict}
\begin{block}{Mutual Exclusion in Distributed Systems} 
Mutual exclusion in distributed systems refers to the challenge of properly accessing a shared resource at any given time by different concurrent processes on different machines. The motivation is ensuring that only one process holds the resource and this resource is orderly accessed for all participants in the system by using coordination and synchronization systems.
\end{block}
\note{}
\end{frame}

\section{Our Motivation}
\begin{frame}
\frametitle{Our Motivation}
\framesubtitle{Do I really need to know everyone in the room to get the permission?}
Think of a hundred worker company, and there is only one printer.
\begin{itemize}
    \item How would you feel if you have to ask everyone for permission every time you want to use?
    \item How would you feel if you are continuously asked for permission even if you don't use the printer?
    \item How would you feel if someone tries to use the printer and implies that he asked first when it's your turn to use?
    \item Would you want to work in a thousand worker company in these situations?
\end{itemize}
\end{frame}
\note{A 100 worker company, you have to use printer but only one person can use at a single time, imagine that you have to ask everyone whether they will use the printer or not. This would be a waste of time and effort. What about a 1000 worker company, how much time would you spend to get the permission? Even if you are not using the device, do you imagine that 900 people asking you for permission continuously? You would want to scream that's enough!}

\section{The Proposed Solution}
\begin{frame}
\frametitle{The Proposed Solution}
\framesubtitle{}

Raymond's algorithm is based on K-ary tree where the root holds the token. While changing parents is frequent for a node, the algorithm guarantees that message complexity will be $O(logN)$ if the topology is structured properly.

That has advantages in terms of:

\begin{itemize}
\item Message overhead
\item Scalability
\item Adaptability to topology
\item Communication sources
\end{itemize}

\end{frame}


\section{Background Information and Previous Works}


\subsection{Background Information}

\frame{
\frametitle{Background Information}
In distributed systems, mutual exclusion is crucial for preventing race conditions, ensuring that only one process can
access a critical section at any given time.

Requirements for a mutual exclusion algorithm in distributed systems are:
\begin{itemize}
    \item \textbf{No Deadlock:} Ensure processes don’t indefinitely wait for messages.
    \item \textbf{No Starvation:} Every process should have a chance to execute its critical section in finite time.
    \item \textbf{Fairness:} Requests to execute critical sections should be executed in the order they arrive.
    \item \textbf{Fault Tolerance:} The system should recognize failures and continue functioning without disruption.
\end{itemize}
}

\subsection{Previous Works}
\begin{frame}{Previous Works}

Previous works are based on different methods.

\begin{itemize}
    \item \textbf{Token Based Algorithm:} Uses a unique token shared among sites, allowing possession of the token to enter the critical section. Examples include the Suzuki-Kasami Algorithm and Raymond’s Algorithm explained in this presentation.
    \item \textbf{Non-token based approach:} Sites communicate to determine which should execute the critical section next, using timestamps to order requests. Examples include the Ricart-Agrawala Algorithm.
    \item \textbf{Quorum based approach:} Sites request permission from a subset called a quorum, ensuring mutual exclusion through common subsets. Examples include Maekawa’s Algorithm.
\end{itemize}

\end{frame}




\section{The Algorithm}
\subsection{Algorithm Description}
\begin{frame}{Algorithm Description}
Raymond's algorithm is based on K-ary tree shaped topology. Every node has these features:

\begin{itemize}
    \item Single Parent Relationship
    \item FIFO Request Queue
    \item Forwarding Requests
\end{itemize}

\end{frame}


\subsection{Pseudocode}

\begin{frame}
\frametitle{Algorithm Pseudocode Part 1}

\begin{center}
\begin{algorithm}[H]
	\scriptsize
	\def\algorithmlabel{Raymond's}
    \caption{\algorithmlabel\ algorithm}
    \label{alg:raymondsalgorithm}
    \begin{algorithmic}[1]
    	\Implement {\algorithmlabel}{cf} 
    	\Use {List of Channels} {channels} 
    	\Use {Integer} {parent\_index} 
    	\Use {Request Queue} {fifo\_queue}
    	\Use {Boolean} {using\_critical\_section}
        \Use {Boolean} {has\_token}
    	\Events{WantUsingCriticalSection, EndUsingCriticalSection, ReceiveRequestMessage, ReceiveTokenMessage}
	   \Need {}
    \end{algorithmic}
\end{algorithm}
\end{center}

\end{frame}

\begin{frame}
\frametitle{Algorithm Pseudocode Part 2}

\begin{center}
\begin{algorithm}[H]
	\scriptsize
	\def\algorithmlabel{Raymond's}
    \caption{\algorithmlabel\ algorithm}
    \label{alg:raymondsalgorithm}
    \begin{algorithmic}[1]
        \Event {WantUsingCriticalSection}{ }
            \begin{enumerate}
                \item if using\_critical\_section = false then
                \item \quad if has\_token = true then
                \item \quad\quad using\_critical\_section ← true;
                \item \quad else then
                \item \quad\quad if fifo\_queue is empty then
                \item \quad\quad\quad send request message into the channel channels[parent\_index];
                \item \quad\quad end if
                \item \quad\quad push -1 into fifo\_queue;
                \item \quad end if
                \item end if
            \end{enumerate}
    \end{algorithmic}
\end{algorithm}
\end{center}
\end{frame}

\begin{frame}
\frametitle{Algorithm Pseudocode Part 3}

\begin{center}
\begin{algorithm}[H]
	\scriptsize
	\def\algorithmlabel{Raymond's}
    \caption{\algorithmlabel\ algorithm}
    \label{alg:raymondsalgorithm}
    \begin{algorithmic}[1]
        \Event {EndUsingCriticalSection}{ }
            \begin{enumerate}
                \item using\_critical\_section ← false;
                \item if fifo\_queue is not empty then
                \item \quad pop from fifo\_queue into i
                \item \quad parent\_index ← find\_index(channels, i);
                \item \quad send token message into the channel channels[parent\_index];
                \item end if
            \end{enumerate}
    \end{algorithmic}
\end{algorithm}
\end{center}
\end{frame}

\begin{frame}
\frametitle{Algorithm Pseudocode Part 4}

\begin{center}
\begin{algorithm}[H]
	\scriptsize
	\def\algorithmlabel{Raymond's}
    \caption{\algorithmlabel\ algorithm}
    \label{alg:raymondsalgorithm}
    \begin{algorithmic}[1]
        \Event {ReceiveRequestMessage}{ }
            \begin{enumerate}
                \item if using\_critical\_section = false then
                \item \quad if has\_token = true then
                \item \quad\quad parent\_index ← find\_index(channels, i);
                \item \quad\quad send token message into the channel channels[parent\_index];
                \item \quad else then
                \item \quad\quad if fifo\_queue is empty then
                \item \quad\quad\quad send request message into the channel channels[parent\_index];
                \item \quad\quad end if
                \item \quad\quad push i into fifo\_queue;
                \item \quad end if
                \item else then
                \item \quad push i into fifo\_queue;
                \item end if
            \end{enumerate}

    \end{algorithmic}
\end{algorithm}
\end{center}
\end{frame}

\begin{frame}
\frametitle{Algorithm Pseudocode Part 5}

\begin{center}
\begin{algorithm}[H]
	\scriptsize
	\def\algorithmlabel{Raymond's}
    \caption{\algorithmlabel\ algorithm}
    \label{alg:raymondsalgorithm}
    \begin{algorithmic}[1]
        \Event {ReceiveTokenMessage}{ }
            \begin{enumerate}
                \item pop from fifo\_queue into j
                \item if j = -1 then
                \item \quad has\_token ← true;
                \item \quad using\_critical\_section ← true;
                \item else then
                \item \quad parent\_index ← find\_index(channels, j);
                \item \quad send token message into the channel channels[parent\_index];
                \item \quad \quad if fifo\_queue is not empty then
                \item \quad \quad send request message into the channel channels[parent\_index];
                \item \quad end if
                \item end if
            \end{enumerate}

    \end{algorithmic}
\end{algorithm}
\end{center}
\end{frame}

\begin{frame}
\frametitle{Strength of The Algorithm}
\framesubtitle{Correctness}
\begin{itemize}
    \item \textbf{Mutual Exclusion}
    \item \textbf{Deadlock}
    \item \textbf{Starvation}
\end{itemize}
\end{frame}

\begin{frame}
\frametitle{Strength of The Algorithm}
\framesubtitle{Complexity}
\begin{itemize}
    \item \textbf{Memory Complexity:} $O(logN)$
    \item \textbf{Message Complexity:} $\frac{2N}{3}$
\end{itemize}
\end{frame}

\subsection{Main Point 3}
\begin{frame}
\frametitle{Main Point 3}
\framesubtitle{Sketch a Proof of the Crucial Results}
The emphasis is on the word ``sketch''. Give a very high-level description of the proofs, emphasizing the proof structure and the proof techniques used. If the proofs have no structure (in which case it may be assumed that you are not the author of the paper), then you must impose one on them. Gloss over the technical details. It is a good idea to point them out but not to explore them.
\end{frame}


\section{Experimental results/Proofs}

\subsection{Main Result 1}
\begin{frame}
\frametitle{Main Result 1}
\framesubtitle{}
Choose \textbf{just the key results}. They should be important, non-trivial, should give the flavour of the rest of the technical details and should be presentable in a relatively short period of time. Use figures instead of tables instead of text.

Better to present 10\% the entire audience gets than 90\% nobody gets
\end{frame}


\subsection{Main Result 2}
\begin{frame}
\frametitle{Main Result 2}
\framesubtitle{Try a subtitle}
\begin{itemize}
\item Make sure your notation is clear and consistent throughout the talk. Prepare a slide that explains the notation in detail, in case that is needed or if somebody asks.
\item Always label all of your axes on graphs; use short but helpful captions on figures and tables. It is also very useful to have an arrow on the side which clearly shows which direction is considered better (e.g., "up is better").
\item If you have experimental results, make sure you clearly present the experimental paradigm you used, and the details of your methods, including the number of trials, the specific analysis tools you applied, significance testing, etc.
\item The talk should contain at least a brief discussion of the limitations and weaknesses of the presented approach or results, in addition to their strengths. This, however, should be done in an objective manner -- don't enthusiastically put down your own work.
\end{itemize}
\end{frame}


\subsection{Main Result 3}
\begin{frame}
\frametitle{Main Result 3}
\framesubtitle{}
\begin{itemize}
\item If time allows, the results should be compared to the most related work in the field. You should at least prepare one slide with a summary of the related work, even if you do not get a chance to discuss it. This will be helpful if someone asks about it, and will demonstrate your mastery of the material.
\item Spell check again.
\item Give for each of the x-axis, y-axis, and z-axis
\item Label, unit, scale (if log scale)
\item Give the legend
\item Explain all symbols
\item Take an example to illustrate a specific point in the figure
\end{itemize}
\end{frame}



\section{Conclusions}
\begin{frame}
\frametitle{Conclusions}
\framesubtitle{Hindsight is Clearer than Foresight}
Advices come from \cite{spillman2000present}.
\begin{itemize}
\item You can now make observations that would have been confusing if they were introduced earlier. Use this opportunity to refer to statements that you have made in the previous three sections and weave them into a coherent synopsis. You will regain the attention of the non- experts, who probably didn’t follow all of the Technicalities section. Leave them feeling that they have learned something nonetheless.
\item Give Open Problems It is traditional to end with a list of open problems that arise from your paper. Mention weaknesses of your paper, possible generalizations, and indications of whether they will be fruitful or not. This way you may defuse antagonistic questions during question time.
\item Indicate that your Talk is Over
An acceptable way to do this is to say “Thank-you. Are there any questions?”\cite{einstein}
\end{itemize}

\end{frame}

\section*{References}
\begin{frame}{References}
\tiny
\bibliographystyle{IEEEtran}
\bibliography{refs}
\end{frame}

\begin{frame}{How to prepare the talk?}
Please read \url{http://larc.unt.edu/ian/pubs/speaker.pdf}
\begin{itemize}
\item The Introduction:  Define the Problem,    Motivate the Audience,    Introduce Terminology,    Discuss Earlier Work,    Emphasize the Contributions of your Paper,    Provide a Road-map.
\item The Body:    Abstract the Major Results, Explain the Significance of the Results, Sketch a Proof of the Crucial Results
\item Technicalities: Present a Key Lemma, Present it Carefully
\item The Conclusion: Hindsight is Clearer than Foresight, Give Open Problems, Indicate that your Talk is Over
\end{itemize}

\note{
\begin{itemize}
\item The Introduction:  Define the Problem,    Motivate the Audience,    Introduce Terminology,    Discuss Earlier Work,    Emphasize the Contributions of your Paper,    Provide a Road-map.
\item The Body:    Abstract the Major Results, Explain the Significance of the Results, Sketch a Proof of the Crucial Results
\item Technicalities: Present a Key Lemma, Present it Carefully
\item The Conclusion: Hindsight is Clearer than Foresight, Give Open Problems, Indicate that your Talk is Over 
\end{itemize}
}
\end{frame}



\thankslide




\end{document}